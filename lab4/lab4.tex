\documentclass[bachelor, och, labwork]{shiza}

\usepackage[utf8]{inputenc}
\usepackage{graphicx}

\usepackage[sort,compress]{cite}
\usepackage{amsmath}
\usepackage{amssymb}
\usepackage{amsthm}
\usepackage{fancyvrb}
\usepackage{longtable}
\usepackage{array}
\usepackage[english,russian]{babel}
\usepackage{minted}

\usepackage{tempora}


% \usepackage[colorlinks=false]{hyperref}


\newcommand{\eqdef}{\stackrel {\rm def}{=}}


\begin{document}

\title{}

\course{4}

\group{431}

\napravlenie{10.05.01 "--- Компьютерная безопасность}


\author{Никитина Арсения Владимировича}


\satitle{доцент}
\saname{А.\,В.\,Жаркова}


\date{2022}

\maketitle

% Включение нумерации рисунков, формул и таблиц по разделам
% (по умолчанию - нумерация сквозная)
% (допускается оба вида нумерации)
%\secNumbering


\tableofcontents





\section{Задание лабораторной работы}
Пусть задан мультипликативный конгруэнтный генератор:

\begin{center}
$x_{t+1} = \alpha x_t(mod ~n)$
\end{center}

Предположим, что НОД($x_0, ~n$) = 1. Каково возможное максимальное значение 
периода выпускной последовательности?


\section{Теоретическая часть}

\subsection{Лемма о разложении числа на простые множители}

Пусть число $n$ допускает разложение на простые множители в виде:

\begin{center} $n=p_1^{e_1}\times p_2^{e_2}...\times p_k^{e_k}$\end{center}.

Длина периода мультипликативной конгруэнтной последовательности, определенной
параметрами ($x_0$, $\alpha$, $m$), является наименьшим общим кратным длин периодов
последовательностей ($x_0 ~mod~p_j^{e_j}, ~\alpha_0 ~mod~p_j^{e_j}, ~p_j^{e_j}$).


Для начала рассмотрим любое число $n$. Оно может быть как простым так и нет.

\subsection{Случай 1. Число n не является простым}
Рассмотрим ситуацию, когда $n$ --- составное число. Итак, если число $n$ имеет
делитель $d$, и если $x_t$ , будет кратно этому числу $d$, то все последующие
элементы последовательности начиная с $x_{t+1}$, то есть последовательность будет
вырождаться в 0 начиная с этого числа. Так что нам необходимо, чтобы не только
начальный элемент последовательности ($x_0$) был взаимнопрост с $n$, но и все 
последующие, что и будет ограничивать период последовательности до значения
функции Эйлера от числа $n$, то есть до количества взаимнопростых с числом
$n$ чисел в промежутке от 1 до $n-1$.

Согласно приведенной лемме период последовательности зависит исключительно от
периодов последовательностей при $n=p^e$

Итак, если $x_t=\alpha^{t}x_0~mod~p^e$ и ясно, что период будет иметь длину 1,
если $\alpha$ будет кратно $p$. Поэтому будем считать, что $a$ и $p$ взаимнопростые.
Тогда период будет наименьшим целым числом $\lambda$, таким, что $x_0=\alpha^\lambda x_0~mod~p^e$.
Если НОД($x_0,p^e$) = $p^f$, то это можно переписать как:

\begin{center}$\alpha^\lambda \equiv 1 (mod ~p^{e-f})$\end{center}. Поэтому
$\lambda$ является делителем $\varphi(p^{e-f})=p^{e-f-1}(p-1)$.

Когда $\alpha$ и $m$ --- взаимнопростые числа, наименьшее число $\lambda$, для
которого $\alpha^\lambda \equiv 1 ~(mod~n)$, принято называть порядком
$\alpha$ по модулю $n$. Любое такое значение $\alpha$, которое имеет максимальный
возможный порядок по модулю $m$, называется первообразным элементом по модулю $m$.

Обозначим через $\lambda(n)$ порядок первообразного элемента. Так как $p^{e-1}(p-1) ~\vdots~ \lambda(p^e)$,
то можно определить и порядок $m$ с помощью следующих соотношений:

\begin{center}

$\lambda(2) = 1,$

$\lambda(4) = 2,$

$\lambda(2^e)=2^{e-2}$, если $e \geq 3,$

$\lambda(p^e)=p^{e-1} (p-1)$, если $e > 2,$

$\lambda(p_1^{e_1} \times p_2^{e_2}...\times p_k^{e_k})= \text{НОК}(\lambda(p_1^{e_1}),...,\lambda(p_k^{e_k}))$, если $e > 2$.
\end{center}

\subsection{Случай 2. Число n простое}

Так как мы выяснили, что длина периода ограничена значением функции Эйлера, то можем
получить, что для простого числа значение функции Эйлера максимально и равно количеству
взаимно простых с ним, то есть $n-1$.

\subsection{Объединение случаев}

Так как по условию имеем $x_0$ и $n$ --- взаимнопросты, то длина периода будет
зависеть от степени перовобразного элемента в общем случае и не будет зависеть от
этого, когда само число $n$ является простым и максимальный период для данного
числа известен и равен $n-1$.


\end{document}